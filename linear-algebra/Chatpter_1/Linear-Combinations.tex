\documentclass{article}
\usepackage{amsmath}
\usepackage{amssymb}
\usepackage{amsfonts}
\usepackage{array}
\usepackage[margin=0.5in]{geometry}  % Set all margins to 0.5 inch

% Remove extra space around section headers
\usepackage{titlesec}
\titlespacing*{\section}{0pt}{12pt}{6pt}
\titlespacing*{\subsection}{0pt}{12pt}{6pt}

% Adjust equation spacing
\setlength{\abovedisplayskip}{6pt}
\setlength{\belowdisplayskip}{6pt}
\setlength{\abovedisplayshortskip}{3pt}
\setlength{\belowdisplayshortskip}{3pt}


\begin{document}

\section{Linear Combinations}

\subsection{Definition}
A linear combination of vectors $\vec{v_1}, \vec{v_2}, \ldots, \vec{v_n}$ is a vector $\vec{y}$ defined by:
\[
\vec{y} = c_1\vec{v_1} + c_2\vec{v_2} + \cdots + c_n\vec{v_n}
\]
where $c_i$ are scalars called weights.

\subsection{Example 1}
Given vectors:
\[
\vec{v_1} = \begin{bmatrix} 1 \\ 5 \end{bmatrix} \text{ and }
\vec{v_2} = \begin{bmatrix} -3 \\ 4 \end{bmatrix}
\]

We can form linear combinations such as:
\[
\vec{u} = 2\vec{v_1} - \vec{v_2} \text{ and }
\vec{w} = \frac{1}{4}\vec{v_1} + \frac{5}{3}\vec{v_2}
\]

\section{Linear Combinations Existence}

\subsection{Determining Existence}
To determine if a vector $\vec{b}$ can be written as a linear combination of given vectors, we solve the system:
\[
c_1\vec{v_1} + c_2\vec{v_2} = \vec{b}
\]

\subsection{Example 2}
Given:
\[
\vec{v_1} = \begin{bmatrix} -2 \\ -5 \end{bmatrix}, 
\vec{v_2} = \begin{bmatrix} 2 \\ 6 \end{bmatrix}, 
\vec{b} = \begin{bmatrix} 7 \\ 4 \end{bmatrix}
\]

To determine if $\vec{b}$ is a linear combination of $\vec{v_1}$ and $\vec{v_2}$:
\[
c_1\begin{bmatrix} -2 \\ -5 \end{bmatrix} + c_2\begin{bmatrix} 2 \\ 6 \end{bmatrix} = \begin{bmatrix} 7 \\ 4 \end{bmatrix}
\]

This can be solved using row reduction of the augmented matrix.

\section{Span}

\subsection{Definition}
For vectors $\vec{v_1}, \vec{v_2}, \ldots, \vec{v_p}$ in $\mathbb{R}^n$, their span, denoted as $\text{span}\{\vec{v_1}, \vec{v_2}, \ldots, \vec{v_p}\}$, is the set of all possible linear combinations of these vectors.

\subsection{Vector Equation Connection}
A vector equation $x_1\vec{a_1} + x_2\vec{a_2} + \cdots + x_n\vec{a_n} = \vec{b}$ has the same solution set as the linear system whose augmented matrix is $[\vec{a_1} \; \vec{a_2} \; \cdots \; \vec{a_n} \mid \vec{b}]$.

\section{Practice Problem}
A mining company problem demonstrates practical application:

Two mines produce:
\begin{itemize}
    \item Mine 1: 20 tons copper, 550 kg silver per day
    \item Mine 2: 30 tons copper, 500 kg silver per day
\end{itemize}

To produce 150 tons copper and 2825 kg silver:
\[
x_1\begin{bmatrix} 20 \\ 550 \end{bmatrix} + x_2\begin{bmatrix} 30 \\ 500 \end{bmatrix} = \begin{bmatrix} 150 \\ 2825 \end{bmatrix}
\]

Solution:
\begin{itemize}
    \item Mine 1 should operate 1.5 days
    \item Mine 2 should operate 4 days
\end{itemize}

section{Practice Problem 2}
Consider the vectors:
\[
\vec{a_1} = \begin{bmatrix} 1 \\ 4 \\ -2 \end{bmatrix}, 
\vec{a_2} = \begin{bmatrix} -2 \\ -3 \\ 7 \end{bmatrix}, 
\vec{b} = \begin{bmatrix} 4 \\ 1 \\ h \end{bmatrix}
\]

Find the value(s) of $h$ for which $\vec{b}$ lies in the span of $\vec{a_1}$ and $\vec{a_2}$.

\subsection{Solution}
We solve this by setting up an augmented matrix and performing row reduction:
\[
\begin{bmatrix} 
1 & -2 & 4 \\
4 & -3 & 1 \\
-2 & 7 & h
\end{bmatrix} \sim 
\begin{bmatrix}
1 & -2 & 4 \\
0 & 5 & -15 \\
0 & 3 & 8+h
\end{bmatrix} \sim
\begin{bmatrix}
1 & -2 & 4 \\
0 & 1 & -3 \\
0 & 0 & 17+h
\end{bmatrix}
\]

For $\vec{b}$ to be in the span of $\vec{a_1}$ and $\vec{a_2}$, the system must be consistent. This requires:
\[
17 + h = 0
\]

Therefore:
\[
h = -17
\]

This means $\vec{b}$ is in the span of $\vec{a_1}$ and $\vec{a_2}$ if and only if $h = -17$.

\end{document}