\documentclass{article}
\usepackage{amsmath}
\usepackage{array}
\usepackage[margin=0.5in]{geometry}  % Set all margins to 0.5 inch

% Remove extra space around section headers
\usepackage{titlesec}
\titlespacing*{\section}{0pt}{12pt}{6pt}
\titlespacing*{\subsection}{0pt}{12pt}{6pt}

% Adjust equation spacing
\setlength{\abovedisplayskip}{6pt}
\setlength{\belowdisplayskip}{6pt}
\setlength{\abovedisplayshortskip}{3pt}
\setlength{\belowdisplayshortskip}{3pt}

\begin{document}
\section*{Echelon Form vs. Reduced Row Echelon Form (RREF)}

\subsection*{Echelon Form (Previously called Triangle Form)}
A matrix is in Echelon Form if:
\begin{enumerate}
    \item All non-zero rows are above all zero rows
    \item Each leading entry of a row is in a column to the right of the leading entry of the row above it
    \item All entries in a column below a leading entry are zeros
\end{enumerate}

\subsection*{Reduced Row Echelon Form (RREF)}
A matrix is in RREF if it satisfies all conditions of Echelon Form and:
\begin{enumerate}
    \item The leading entry in each non-zero row is 1
    \item Each leading 1 is the only non-zero entry in its column
\end{enumerate}

\subsection*{Pivot Terminology}
\begin{itemize}
    \item \textbf{Pivot Position:} Corresponds to leading 1 in RREF
    \item \textbf{Pivot Column:} The column that contains the pivot
    \item \textbf{Pivot:} Nonzero number in pivot position used to create zeros in row operations
\end{itemize}

\section*{The Row Reduction Algorithm}

The following steps describe the process of reducing a matrix to Reduced Row Echelon Form (RREF):

\begin{enumerate}
    \item Begin at leftmost nonzero column, which is a pivot column. Select a nonzero entry as pivot and interchange, if necessary, to move that entry into the pivot position (Row 1).
    
    \item Use row operations to create zeros in all entries below the pivot.
    
    \item Repeat this process for remaining rows, ignoring rows you've already applied algorithm to.
    
    \item Ensure each pivot is a 1, using scaling operations as necessary.
    
    \item Beginning with the rightmost pivot and working upwards and to the left, use row operations to create zeros above each pivot.
\end{enumerate}

Example working matrix:
\[
\begin{bmatrix}
1 & 0 & -3 & | & 8 \\
0 & 1 & \frac{15}{2} & | & -\frac{9}{2} \\
0 & 0 & 1 & | & -1
\end{bmatrix}
\]

This systematic process transforms the matrix into RREF, making the system easier to solve.

\end{document}