\documentclass{article}
\usepackage{amsmath}
\usepackage{amsthm}
\usepackage{amssymb}
\usepackage{graphicx}

\title{Vectors in $\mathbb{R}^2$}
\author{Notes}
\date{}

\begin{document}
\maketitle

\section{Basic Definitions}

\subsection{Vectors}
A vector is an ordered list of numbers. In $\mathbb{R}^2$, vectors are represented as ordered pairs of real numbers.

\subsection{Column Vectors}
A column vector is a vector with only one column. These are commonly used to represent ordered pairs, triples, etc. In $\mathbb{R}^2$, a column vector has the form:
\[
\vec{u} = \begin{bmatrix} u_1 \\ u_2 \end{bmatrix}
\]

\section{Vector Operations}

\subsection{Basic Operations}
For vectors in $\mathbb{R}^2$, we have the following operations:

\begin{enumerate}
    \item \textbf{Scalar Multiplication:} Multiply vector by a constant
    \[
    c\vec{u} = \begin{bmatrix} cu_1 \\ cu_2 \end{bmatrix}
    \]
    
    \item \textbf{Addition:} Add corresponding values
    \[
    \vec{u} + \vec{v} = \begin{bmatrix} u_1 + v_1 \\ u_2 + v_2 \end{bmatrix}
    \]
    
    \item \textbf{Multiplication:} Not defined for vectors (dimensions don't work)
\end{enumerate}

\section{Special Vectors}

\subsection{Zero Vector}
The zero vector, denoted by $\vec{0}$, is the vector whose entries are all 0:
\[
\vec{0} = \begin{bmatrix} 0 \\ 0 \end{bmatrix}
\]

\section{Algebraic Properties}
For vectors $\vec{u}$, $\vec{v}$, $\vec{w}$ in $\mathbb{R}^n$ and scalars $c$, $d$:

\begin{enumerate}
    \item \textbf{Commutativity of Addition:}
    \[
    \vec{u} + \vec{v} = \vec{v} + \vec{u}
    \]
    
    \item \textbf{Associativity of Addition:}
    \[
    (\vec{u} + \vec{v}) + \vec{w} = \vec{u} + (\vec{v} + \vec{w})
    \]
    
    \item \textbf{Additive Identity:}
    \[
    \vec{u} + \vec{0} = \vec{0} + \vec{u} = \vec{u}
    \]
    
    \item \textbf{Additive Inverse:}
    \[
    \vec{u} + (-\vec{u}) = (-\vec{u}) + \vec{u} = \vec{0}
    \]
    
    \item \textbf{Distributive Properties:}
    \[
    c(\vec{u} + \vec{v}) = c\vec{u} + c\vec{v}
    \]
    \[
    (c + d)\vec{u} = c\vec{u} + d\vec{u}
    \]
    
    \item \textbf{Scalar Multiplication Properties:}
    \[
    c(d\vec{u}) = (cd)\vec{u}
    \]
    \[
    1\vec{u} = \vec{u}
    \]
\end{enumerate}

\section{Example}
Given vectors $\vec{u} = \begin{bmatrix} 2 \\ 3 \end{bmatrix}$ and $\vec{v} = \begin{bmatrix} -1 \\ 2 \end{bmatrix}$, we can perform operations:

\begin{align*}
\vec{u} + \vec{v} &= \begin{bmatrix} 2 \\ 3 \end{bmatrix} + \begin{bmatrix} -1 \\ 2 \end{bmatrix} = \begin{bmatrix} 1 \\ 5 \end{bmatrix} \\
2\vec{u} &= 2\begin{bmatrix} 2 \\ 3 \end{bmatrix} = \begin{bmatrix} 4 \\ 6 \end{bmatrix}
\end{align*}

These operations can be visualized geometrically on a coordinate plane, where vectors are represented as arrows from the origin to their terminal point.

\end{document}